\documentclass[aps,prl,twocolumn,showpacs,superscriptaddress,groupedaddress]{revtex4-2}  % for review and submission
%\documentclass[aps,preprint,showpacs,superscriptaddress,groupedaddress]{revtex4-1}  % for double-spaced preprint

\usepackage{dcolumn}   % needed for some tables
\usepackage{bm}        % for math
\usepackage{amssymb}   % for math
\usepackage{mathtools}  %mathtools is better than amsmath
%\usepackage{color}
\usepackage[pdftex]{graphicx}
\usepackage[svgnames, table]{xcolor}
\usepackage[colorlinks=true, citecolor=Blue,
            linkcolor=BrickRed, urlcolor=ForestGreen]{hyperref}
%\usepackage{mathptmx} %(obsolete)
\usepackage{txfonts}
\usepackage{mathrsfs}

\newcommand{\R}[1]{\textcolor{BrickRed}{#1}}
\newcommand{\B}[1]{\textcolor{blue}{#1}}

\begin{document}



\title{Terrestrial Generation of Gravitational Radiation}
\author{Rana X Adhikari}
\affiliation{Division of Physics, Math, and Astronomy,
California Institute of Technology, Pasadena,
CA 91125, USA}
\date{\today}


\begin{abstract}
All gravitational-wave (GW) detectors are designed for the detection of astrophysical
signals.
The consensus in the GW community is that it is practically impossible to make a signal large enough to detect within the limits of current technology.
Here I show that a phased array can be constructed to make a signal which is detectable on the earth using current GW detection technology.
Terrestrial generation and detection of GWs can be used to test theories of gravity and for non-EM communication.
\end{abstract}


\maketitle


\begin{enumerate}
  \item everyone says you cant make the g-waves: here's why
  \item It would be great if we could make the G-waves: here's why
  \item Radiation of a dumbell
  \item Antenna pattern of a single dumbell
  \item How to calculate the array's antenna pattern
  \item Strain amplitude at 10,000 km
  \item Strain amplitude at 1 kpc
  \item Issues about making the dumbells: failure analysis
  \item Communication bandwidth: PM, AM, or something Hedy Lamarr?
\end{enumerate}


Having a way to generate and detect gravitational waves would be great.
We could test how the radiation works and if its really like GR.

A bunch of people have thought about how to measure GWs in the past. Here is
a list of those with some short descriptions:
\begin{enumerate}
\item dumbells
\item atomic bombs
\end{enumerate}

\paragraph{Dumbells}
Assuming we use a standard quadrupole radiator (a bar of length $L$ with
a sphere of mass $M$ on each end), the far-field strain will be
\begin{equation}
h = \frac{2 G}{c^4 r} \ddot{I}
\end{equation}
as measured a distance $r$ away from the radiator.
Plugging in some usual numbers for the physical dimensions and the rotation
frequency...
we can see that the measured strain would be way to small to detect with
anything we have today (the Advanced LIGO design sensitivity is
$\sim 10^{-24} / \sqrt{Hz}$).

\paragraph{Phased Array}
Phased arrays of electromagnetic radiators can be used to produce focused beams
of radiation (e.g. the Very Large Array, the DSN, etc.) or to receive very weak
signals.
This is how phased arrays work...

Here are some plots of the antenna pattern from having a N x N array of
rotating bars:

FIGURE


\paragraph{Conclusion}
So, as you can see, it would be hard and very expensive, but it can be done.


{\it Acknowledgements.---}
I would like to thank Dan Kennefick for initial discussions of this
concept and to the students from Caltech's spring 2018 class on
gravitational waves for much fruitful discussions.

%%%%%% References %%%%%%
\bibliography{RanaGWrefs}


\end{document}
%
% ****** End of file template.aps ******
